\documentclass[a4paper]{article}

\usepackage[pdftex]{graphicx} % Required for including pictures
\usepackage[serbianc]{babel}
\usepackage[pdftex,linkcolor=black,pdfborder={0 0 0}]{hyperref} % Format links for pdf
\usepackage{calc} % To reset the counter in the document after title page
\usepackage{enumitem} % Includes lists
\linespread{1.2} % Set linespace
\usepackage[a4paper, lmargin=0.1\paperwidth, rmargin=0.1\paperwidth, tmargin=0.1\paperheight, bmargin=0.1\paperheight]{geometry} %margins
%\usepackage{parskip}

\hypersetup{unicode}
\makeindex
\large

\usepackage[all]{nowidow} % Tries to remove widows
%\usepackage[protrusion=true,expansion=true]{microtype} % Improves typography, load after fontpackage is selected

\title{Интегрисани информациони систем за издавање диплома и конкурс за упис на високошколске установе еФакултет}
\author{Борисав Живановић}

\begin{document}

\maketitle

\section{Сажетак}

У раду је описан интегрисани информациони систем за издавање диплома и конкурс за упис на високошколске установе еФакултет.
Применом овог решења омогућено је централизовано издавње и праћење издатих диплома акредитованих вискошколских установа.
Конкурс за упис се ослања на централизоване регистре диплома средњих школа и високошколских установа.

\section{Кључне речи}

информациони систем, високо образовање, факултет, диплома, еУправа

\section{Увод}

Конкурисање за упис на високошколске установе захтева подношење дипломе претходног нивоа образовања.
Такође, конкурисање за посао често захтева доказ о завршеном степену школовања за одређену професију.
Традиционални приступ подношења папирних диплома је подложан злоупотреби, како због подношења фалсификованих диплома,
тако и због немогућности праћења издатих диплома. Аутоматизовано и централизовано праћење података о издатим дипломама
омогућава већу транспарентност у односу на традиционални приступ и олакшава праћење издатих диплома.

\section{Сродна истраживања}

Већина факултета има информациони систем студентске службе у којем се чувају подаци о издатим дипломама матичног факултета.
Мањи број факултета (попут Факултета техичких наука Универзитета у Новом Саду) имају информациони систем за конкурс, али
као доказ о завршеном претходном нивоу школовања се подносе скениране папирне дипломе. Такав приступ захтева ручну проверу
исправности диплома и повећава вероватноћу за грешком (погрешно унет просек, прихватање фалсификоване дипломе).

\section{Коришћене технологије}

За имплементацију клијентске апликације коришћен је програмски језик TypeScript и окружење (енгл. framework) Angular.
За имплементацију клијентске апликације коришћен је програмски језик Java и окружење Spring, уз библиотеку
за објектно-релационо мапирање Hibernate.
За ауторизацију је коришћен централизовани SSO (Single Sign On) сервер еУправе (модификована верзија протокола OAuth 2.0).
За складиштење података коришћена је MySQL база података.

\section{Спецификација захтева}

Корисници система су администратор и студент. Администратор расписује конкурс (са одговарајућим квотама) и окончава пријаве,
а потом и конкурс. По окончавању конкурса, систем рангира пријаве по просеку, и пријављени из горњег дела ранг листе
(по квоти) добијају могућност за упис. Администратор уноси податке о завршетку студија. Студент се пријављује по конкурсу, уз диплому из регистра. По изласку резултата, прихвата или одбија упис (уколико је распоређен по квоти за упис).

\section{Спецификација дизајна}

Систем је реализован као веб апликација. Клијентска апликација комуницира преко HTTP API-ја са серверском апликацијом. API је
дизајниран у складу са REST архитектонским стилом.

\section{Имплементација}

\section{Демонстрација}

\section{Закључак}

У раду је описан интегрисани информациони систем за издавање диплома и конкурс за упис на високошколске установе еФакултет.
Приказано је како је применом централизованих регистара диплома средњих школа и високошколских установа могуће убразати
конкурс, као и смањити могућност грешки и представљања фалсификованих диплома. У тренутној имплементацији, регистар диплома
и конкурс су обједињени у један систем. У будућности, могуће је раздвајање на две групе система:
централизовани државни регистар диплома и на појединачне системе за конкурс (прилагођене специфичним потребама и
правилима конкурса). Могућ правац развоја би био омогућавање јавног приступа подацима о завршеним степенима образовања
државних функционера, службеника као и осталих лица од важности (лекара, наставника). За имплементацију такве функционалности
би било потребно доношење додатних законских оквира.

\end{document}
